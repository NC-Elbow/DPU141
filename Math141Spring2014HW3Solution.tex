\documentclass[10 pt]{amsart}
\usepackage{amscd,amsmath,amsthm,amssymb}
\usepackage{enumerate,varioref}
\usepackage{epsfig}
\usepackage{graphicx}
\usepackage{mathtools}
\newtheorem{thm}{Theorem}
\newtheorem{cor}[thm]{Corollary}
\newtheorem{lem}[thm]{Lemma}
\newtheorem{prop}[thm]{Proposition}
\theoremstyle{definition}
\newtheorem{defn}[thm]{Definition}
\theoremstyle{remark}
\newtheorem{ex}[thm]{Example}
\newtheorem{rem}[thm]{Remark}
\numberwithin{equation}{subsection}
\newcommand{\R}{\mathbb{R}}
\newcommand{\Z}{\mathbb{Z}}
\newcommand{\C}{\mathbb{C}}
\newcommand{\Q}{\mathbb{Q}}
\newcommand{\lh}{\lim_{h\rightarrow 0}}
\begin{document}

\title{Homework 3 Maths 141 Spring 2014}
\maketitle 

5.6.28\\
Prove that $F_{k+1}^2 - F_k^2 - F_{k-1}^2 =2F_k F_{k-1}$ for all $k\geq1$ where $\{F_n\}$ is the Fibonacci sequence.\\

Solution:\\
We know that $F_{k+1} = F_k + F_{k-1}$.  So substituting in the equation we see
\begin{eqnarray*}
F_{k+1}^2 - F_k^2 - F_{k-1}^2 & = & (F_k + F_{k-1})^2 - F_k^2 - F_{k-1}^2\\
&=& F_k^2 + 2F_k F_{k-1} + F_{k-1}^2 - F_k^2 - F_{k-1}^2\\
&=& 2F_k F_{k-1}.
\end{eqnarray*} 

\newpage
5.6.33\\
It turns out that the Fibonacci sequence satisfies the following explicit formula: For all integers $n\geq0$
\[
F_n = \frac{1}{\sqrt{5}}\left[\left(\frac{1+\sqrt{5}}{2}\right)^{n+1}- \left(\frac{1-\sqrt{5}}{2}\right)^{n+1} \right]
\]

Solution:\\
This is a matter of plugging numbers in directly and simplifying.  Let's make matters easier on ourselves and make the definitions:
\begin{eqnarray*}
r_1 &=& \left(\frac{1+\sqrt{5}}{2}\right),\\
r_2 &=&  \left(\frac{1-\sqrt{5}}{2}\right).
\end{eqnarray*}

As we saw in class, and is a well known fact both of these $r_i$ satisfy the quadratic equation
\[
r^2 - r-1=0.
\]
Now we substitute $F_n = A(r_1^{n+1}-r_2^{n+1})$ and begin our computation.\\
\begin{eqnarray*}
F_{n+1} -F_n - F_{n-1} &=& A(r_1^{n+2}-r_2^{n+2}) -A(r_1^{n+1}-r_2^{n+1}) -A(r_1^{n}-r_2^{n})\\
&=& A ((r_1^{n+2} - r_1^{n+1}-r_1^n) - (r_2^{n+2} - r_2^{n+1}-r_2^n))\\
&=& A(r_1^n(r_1^2-r_1-1) - r_2^n(r_2^2-r_2-1))\\
&=& A(r_1^n\cdot(0)-r_2^n\cdot(0))\\
&=&0.
\end{eqnarray*}



\newpage
5.714\\
The following sequence is defined recursively.  Use iteration to give an explicit formula for the sequence.  Simplify if possible.
\[
x_{k} = 3x_{k-1}+k
\]
$x_1=1$.

Solution:\\
Let's not worry about the value of$x_1$ just yet and see if we can simply find a pattern in the iterations.
\begin{eqnarray*}
x_2 & = & 3x_1+2\\
x_3 &=& 3x_2+3\\
&=& 3(3x_1+2)+3\\
&=& 3^2x_1 + 3^1\cdot 2 + 3\\
x_4 &=& 3x_3+4\\
&=& 3(3^2x_1 + 3^1\cdot 2 + 3)+4\\
&=& 3^3 x_1 + 3^2\cdot 2+ 3\cdot 3 + 4\\
x_5 &=& 3x_4+5\\
&=& 3(3^3 x_1 + 3^2\cdot 2+ 3\cdot 3 + 4)+5\\
&=& 3^4x_1+3^3\cdot2+3^2\cdot 3+3\cdot 4 +5
\end{eqnarray*}

What appears to be happening is that for $x_k$ the powers of 3 decrease on each successive multiplier.  So it appears that we have something of the form
\[
x_k = 3^{k-1}x_1 + \sum_{j=2}^{k}3^{k-j}\cdot j
\]
Noting that $x_1=1$ we have
\[
x_k = \sum_{j=1}^{k}3^{k-j}\cdot j.
\]

Let's test this formula against the given recurrence relation.
\begin{eqnarray*}
x_{k+1} &=&  3x_k+(k+1)\\
&=& 3(\sum_{j=1}^{k}3^{k-j}\cdot j)+(k+1)\\
&=& \sum_{j=1}^{k}3^{k+1-j}\cdot j + (k+1)\\
&=& \sum_{j=1}^{k+1}3^{k+1-j}\cdot j.
\end{eqnarray*}
which is exactly what we had expected.
\newpage
5.7.26\\
A person saving for retirement makes an intial deposit of \$1000 to a bank account earning interest at a rate of 3\% per year compounded monthly, and each month she adds an additional \$200 to the account.\\
a. For each nonnegative integer $n$, let $A_n$ be the amount in the account at the end of $n$ months.  Find a recurrence relation relating $A_k$ to $A_{k-1}$.\\
b.Use iteration to find an explicit formula for $A_n$.\\
c. Use mathematical induction to prove the correctness of the formula you obtained in part(b).\\
d. How much will the account be worth in 20 years? In 30 years?\\
e. How long until the account has a value of \$10,000?

Solution:\\
a. This is fairly straightforward.  Interest compounded $n$ times anually pays out an interest rate of $r/n$ per cycle.  In this case then
\[
A_{n+1} = A_n (1+\frac{.03}{12}) + 200.
\]


b. Before solving this particular recurrence relation, let's simplify our notation slightly.  Let $q=1+.03/12$ and $m=200$.
Then we see the relation from part a. reduced to 
\[
A_{n+1}=qA_n + m.
\]
Iterating this a few times (as we did above) we'll see that
\begin{eqnarray*}
A_1 &=& qA_0 + m\\
A_2 &=& qA_1+m\\
&=& q(qA_0+m)+m\\
A_3 &=& qA_2+m \\
&=& q(q^2A_0 + qm +m)+m\\
&=& q^3A_0 + q^2m+qm+m\\
\vdots &&\\
A_n &=& q^nA_0 + m(q^{n-1}+q^{n-2}+\dots+q+1)
\end{eqnarray*}


Now we know from class (and if necessary prove it by induction) of the geometric sum
\[
\sum_{j=0}^{k}r^j = \frac{r^{k+1}-1}{r-1}
\]
Letting our ratio be $q$ we finally see
\[
A_n = q^nA_0 + m\cdot\frac{q^n-1}{q-1}
\]
We can fill in the numbers $q=1.0025$ and $m=200$ whenever we like.

c. Prove the formula derived in part (b) by mathematical induction.
\begin{lem}
The sequence
\[
A_n = q^nA_0 + m\cdot\frac{q^n-1}{q-1}
\]
Satisfies the recurrence relation
\[
A_{n+1} = qA_n +m
\]
\end{lem}
\begin{proof}
We shall proceed by mathematical induction.  In the base case $n=1$ we see. 
\begin{eqnarray*}
A_1 &=& qA_0+m\\
&=& q^1A_0 + m\cdot\frac{q^1-1}{q-1}
\end{eqnarray*}
This is simply a direct computation.
Now for the inductive step.
Assume for some value $k$ we have that
\[
A_k =q^kA_0 + m\cdot\frac{q^k-1}{q-1}
\]
satisfies the reccurence relation as above.  Let's show that
\[
A_{k+1} =q^{k+1}A_0 + m\cdot\frac{q^{k+1}-1}{q-1}
\]
satisfies as well.

\begin{eqnarray*}
A_{k+1} &=& qA_k+m \\
&=& q(q^kA_0 + m\cdot\frac{q^k-1}{q-1})+m\\
&=& q^{k+1} A_0 + qm\cdot\frac{q^k-1}{q-1}+qm\\
&=&q^{k+1} A_0 + qm\cdot\frac{q^k-1}{q-1}+qm\frac{q-1}{q-1}\\
&=&q^{k+1}A_0 + m\frac{q^{k+1}-q+q-1}{q-1}\\
&=& q^{k+1}A_0 + m\cdot\frac{q^{k+1}-1}{q-1}
\end{eqnarray*}
Thus the sequence satisfies the relation at the $k+1^{st}$ term if it satisfies at the $K^{th}$ term.
\end{proof}

d. This is a simple matter of asking how months are 20 and 30 years respectively.
For the 20 year solution $n=240$ so
\[
A_{240} = q^{240}A_0 + m\frac{q^{240}-1}{q-1}
\]
We can plug in the numbers as necessary.
For the 30 year case
\[
A_{360} = q^{360}A_0 + m\frac{q^{360}-1}{q-1}
\]

e. For the time at which we will achieve \$10,000 in the account we simply solve for $n$ in
\[
10000 = q^nA_0 + m\frac{q^n-1}{q-1}
\]
This is a matter of algebra.

\newpage
5.7.53\\
Compute $\begin{bmatrix}
1&1\\
1&0
\end{bmatrix}^n$ for small values of $n$ (up to about 5 or 6).  Conjecture an explicit formula for each entry in the matrix and prove your conjecture using mathematical induction. 

The matrices for the first few iterations are
\[
\begin{bmatrix}
1&1\\
1&0
\end{bmatrix},
\begin{bmatrix}
2&1\\
1&1
\end{bmatrix},
\begin{bmatrix}
3&2\\
2&1
\end{bmatrix},
\begin{bmatrix}
5&3\\
3&2
\end{bmatrix}
\]
These look like the Fibonacci numbers arranged in the following way
\[
\begin{bmatrix}
F_{n+1}&F_n\\
F_n&F_{n-1}
\end{bmatrix}
\]
Let's prove this.
\begin{lem}
Let $\{F_n\}$ be the Fibonacci sequence with $F_0=0$ and $F_1=1$.
Then 
\[
\begin{bmatrix}
1&1\\
1&0
\end{bmatrix}^n = 
\begin{bmatrix}
F_{n+1}&F_n\\
F_n&f_{n-1}
\end{bmatrix}
\]
\end{lem}

\begin{proof}
We begin by induction.  The base step is easy
\[
\begin{bmatrix}
1&1\\
1&0
\end{bmatrix}=\begin{bmatrix}
F_2&F_1\\
F_1&F_0
\end{bmatrix}
\]

Now let's look at the inductive step.  Assuming this is true for some value $k$.  Let's show the $k+1^{st}$ step to be true.
\begin{eqnarray*}
\begin{bmatrix}
1&1\\
1&0
\end{bmatrix}^{k+1} &=& 
\begin{bmatrix}
1&1\\
1&0
\end{bmatrix}^k * 
\begin{bmatrix}
1&1\\
1&0
\end{bmatrix}\\
&=& \begin{bmatrix}
F_{k+1}&F_k\\
F_k&F_{k-1}
\end{bmatrix}*
\begin{bmatrix}
1&1\\
1&0
\end{bmatrix}\\
&=& \begin{bmatrix}
F_{k+1}+F_k& F_{k+1}\\
F_{k}+F_{k-1} & F_k
\end{bmatrix}\\
&=&\begin{bmatrix}
F_{k+2}&F_{k+1}\\
F_{k+1}&F_{k}
\end{bmatrix}
\end{eqnarray*}
\end{proof}

\end{document}