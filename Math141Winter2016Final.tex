\documentclass[16 pt]{amsart}
\usepackage{amscd,amsmath,amsthm,amssymb}
\usepackage{enumerate,varioref}
\usepackage{epsfig}
\usepackage{graphicx}
\usepackage{mathtools}
\newtheorem{thm}{Theorem}
\newtheorem{cor}[thm]{Corollary}
\newtheorem{lem}[thm]{Lemma}
\newtheorem{prop}[thm]{Proposition}
\theoremstyle{definition}
\newtheorem{defn}[thm]{Definition}
\theoremstyle{remark}
\newtheorem{ex}[thm]{Example}
\newtheorem{rem}[thm]{Remark}
\numberwithin{equation}{subsection}
\newcommand{\R}{\mathbb{R}}
\newcommand{\Z}{\mathbb{Z}}
\newcommand{\C}{\mathbb{C}}
\newcommand{\Q}{\mathbb{Q}}
\newcommand{\lh}{\lim_{h\rightarrow 0}}
\begin{document}

\title{Final Exam Maths 141 Winter 2015 \\ DePaul University\\Dr. Alexander}
\maketitle
You have 135 minutes to complete this exam.  Calculators are allowed, but no other electronic devices are permitted.  Please write all your answers in complete, legible sentences, and show all your work to receive full credit.  There are six (6) problems here.  Please solve all of them.  
\vspace{1in}


%table
\begin{center}
  \begin{tabular}{ c | c }
    Problem & Score\\
    \hline
    &\\
    1&\\
    &\\
    2&\\
    &\\
    3&\\
    &\\
    4&\\
    &\\
    5&\\
    &\\
    6&\\
    &\\
    Bonus Round 
    &\\
    \hline 
    &\\    
    Total & 
 \end{tabular}
\end{center}


\newpage


Problem 1: Negate the following statement
\[
\forall x\in \Z, \exists y\in\R \text{ such that, If } x > 4 \text{ then } x^2 > y.
\]

 
\newpage


Problem 2: Prove by induction:  $\forall n>0$
\[
\sum_{j=0}^{n} 3^j = \frac{3^{n+1}-1}{2}
\]

\newpage

Problem 3: Prove by induction:
\[
\begin{bmatrix}
a & b \\
0 & a
\end{bmatrix}^n = 
\begin{bmatrix}
a^n & nba^{n-1} \\
0 & a^n
\end{bmatrix}
\]
Where $a,b\in \R$ and $n\ge 1$.

\newpage

Problem 4: Let $A= \Z \times \Z$.  Define a relation $R$ on $A$ as follows:
\[
(a,b) R (c,d) \iff a + d = c + b.
\]
a. Prove this is an equivalence relation.\\
b. Describe the equivalence classes of $R$.

\vspace{.5in}

Hint: Order Matters Here! Think about the equivalence classes for rationals...

\newpage

Problem 5:(a) Find an inverse of 17 $\mod{41}$.\\

(b) Find the smallest positive number $n$ so that
\[
17 n \equiv 14 \mod{41}
\]

\newpage

Problem 6: Show that
\[
2^n \text{ is } O(n!).
\]
 
\newpage

Bonus Round:  Here are your category three questions:

Bonus 1: Show 
\[
\ln(n) < \sum_{j=1}^{n} \frac{1}{j}
\]

\vspace{.75in}

Bonus 2: Consider again, the ``tribonacci sequence"
\[
a_n = a_{n-1}+a_{n-2}+a_{n-3}
\]
Show that $a_n \in O(1.9^n)$ Can you find $b$ so that $a_n \in \Theta(b^n)$?

\vspace{.75in}

Bonus 3: Show
\[
\begin{bmatrix}
0 & 1 \\ -1 & 0
\end{bmatrix}^n =
\begin{bmatrix}
(-1)^{n+1}(\frac{1}{2}+(-1)^{n}\frac{1}{2}) & (-1)^{n+1}(\frac{1}{2}+(-1)^{n+1}\frac{1}{2})\\
(-1)^{n}(\frac{1}{2}+(-1)^{n+1}\frac{1}{2})
 & (-1)^{n+1}(\frac{1}{2}+(-1)^{n}\frac{1}{2})
\end{bmatrix}
\]

\vspace{.75in}

Bonus 4: Show or give a counterexample

\[
f(x) \in O(g(x)) \implies f(h(x)) \in O(g(h(x)))
\]

\vspace{.75in}

Bonus 5: Show
\[
\sum_{n=1}^{\infty} \frac{n^n}{2^{2^n}} < 1
\]


\vspace{.75in}

Bonus 6: Show
\[
n^{n^2} \in O(2^{2^n})
\]

\end{document}