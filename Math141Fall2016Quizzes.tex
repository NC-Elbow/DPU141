\documentclass{amsart}
\usepackage{amscd,amsmath,amsthm,amssymb}
\usepackage{enumerate,varioref}
\usepackage{epsfig}
\usepackage{graphicx}
\usepackage{mathtools}
\usepackage{tikz}
\usepackage{qcircuit}
\newtheorem{thm}{Theorem}
\newtheorem{cor}[thm]{Corollary}
\newtheorem{lem}[thm]{Lemma}
\newtheorem{prop}[thm]{Proposition}
\theoremstyle{definition}
\newtheorem{defn}[thm]{Definition}
\theoremstyle{remark}
\newtheorem{ex}[thm]{Example}
\newtheorem{rem}[thm]{Remark}
\numberwithin{equation}{section}
\newcommand{\R}{\mathbb{R}}
\newcommand{\Z}{\mathbb{Z}}
\newcommand{\C}{\mathbb{C}}
\newcommand{\Q}{\mathbb{Q}}
\newcommand{\half}{\frac{1}{2}}

\title{Fall 2016 Examples and Quizzes Math 141}
\author{GCA}

\begin{document}
\maketitle

Quiz 1:\\
(1) Prove the following by induction
\[
\forall n>0,\hspace{2pt} \sum_{j=1}^{n}(2j-1) = n^2
\]

In the base case ($n=1$) we have
\[
(2(1)-1) = 1^2
\]
Now we assume that a single one of these equations holds true at $k$
\[
\sum_{j=1}^{k} (2j-1) = k^2
\]
Let's show that it also holds true at $k+1$ using this assumption.

\begin{eqnarray}
\sum_{j=1}^{k+1}(2j-1) & = & \sum_{j=1}^{k}(2j-1) + 2(k+1)-1\nonumber\\
& = & k^2 + (2k+1)\nonumber\\
& = & (k+1)^2.
\end{eqnarray}

So this holds true.

\vspace{1in}

(2) Prove the following by induction
\[
\forall n>0, \hspace{2pt}
\begin{bmatrix}
2 & 0 \\
b & 2
\end{bmatrix}^n = 
\begin{bmatrix}
2^n & 0\\
nb2^{n-1} & 2^n
\end{bmatrix}
\]


Again, in the base case we have $n=1$
\[
\begin{bmatrix}
2 & 0 \\
b & 2
\end{bmatrix}^1 =
\begin{bmatrix}
2 & 0 \\
(1)b2^0 & 2
\end{bmatrix}   
\]
which checks out. Now let's try the inductive case.  Let's assume that this holds at some particular $k$ 
\[
\begin{bmatrix}
2 & 0 \\
b & 2
\end{bmatrix}^k = 
\begin{bmatrix}
2^k & 0\\
kb2^{k-1} & 2^k
\end{bmatrix}
\]

And let's show that it also holds for $k+1$.

\begin{eqnarray}
\begin{bmatrix}
2 & 0 \\
b & 2
\end{bmatrix}^{k+1} & = &
\begin{bmatrix}
2 & 0 \\
b & 2
\end{bmatrix}^k  
\begin{bmatrix}
2 & 0\\
b & 2
\end{bmatrix} \nonumber \\
& = & 
\begin{bmatrix}
2^k & 0\\
kb2^{k-1} & 2^k
\end{bmatrix}
\begin{bmatrix}
2 & 0 \\
b & 2
\end{bmatrix} \nonumber \\
& = &
\begin{bmatrix}
2^{k+1} & 0\\
kb2^{k-1}2 + b2^k & 2^{k+1}
\end{bmatrix}\nonumber
\end{eqnarray}


An examination of the bottom left term shows us
\[
kb2^{k-1}2 + b2^k = (k+1)b2^k
\]

which is exactly what we wanted. 

\vspace{1in}

(3) Prove the following by induction
\[
6|(n^3-n)
\]

In the base case $n=0$ we have $6|0$ which is great since everything divides zero. Now we assume this works for some $k$
\[
6|k^3-k
\]

Now we show this works for $k+1$.  Let's examine
\[
(k+1)^3  - (k+1) = k^3+3k^2+3k+1-k-1 = (k^3-k) + 3k(k+1)
\]

By assumption 
\[
6|(k^3-k)
\]

Now if $6|3k(k+1)$ then we're done since we know
\[
a|b \wedge a|c \implies a|(b+c)
\]

We see 
\[
6|3k(k+1)
\]

since $3k(k+1)$ obviously has a factor of 3.  And one of $k$ or $k+1$ is always even, and thus we have a factor of 2.  So 6 divides this additional piece and therefore
\[
6|(k+1)^3-(k+1)
\]
which is what we wished to show.

\vspace{1in}

(4) Prove DeMoivre's theorem by induction
\[
(\cos(\theta)+ i \sin(\theta))^n = \cos(n\theta) + i \sin(n\theta)
\]

In the base base case $n=1$ there is almost nothing to show

\[
(\cos(\theta)+ i \sin(\theta))^1 = \cos(1\theta) + i \sin(1\theta)
\]


In the inductive case we assume that this works for $k$
\[
(\cos(\theta)+ i \sin(\theta))^k = \cos(k\theta) + i \sin(k\theta)
\]

and show that it works for $k+1$.

\[
(\cos(\theta)+ i \sin(\theta))^{k+1} = (\cos(k\theta) + i \sin(k\theta))(\cos(\theta)+i\sin(\theta))
\]

Expanding and collecting the terms without $i$ and with we have
\[
\cos(k\theta) \cos(\theta) - \sin(k\theta)\sin(\theta)
\]

which we'll recognize as 
\[
\cos((k+1)\theta)
\]


Now for the terms containing $i$ we have
\[
\sin(k\theta)\cos(\theta) + \cos(k\theta)\sin(\theta)
\] 
which we recognize as

\[
\sin((k+1)\theta)
\]


which is exactly what we wished to show.

\vspace{1in}

(5) Let $X_n = \{j| 2\le j \le n\}$.
Let $S = \mathcal{P}(X_n)-\emptyset$.  Let $S_i$ be the elements of $S$ and 
\[
P_i = \prod_{x\in S_i} x
\]

Prove the following by induction:

\[
\sum_{i=1}^{2^{n-1}-1} P_i = \frac{(n+1)!}{2} -1
\]

Before we begin the solution, let's take a look at exactly what we're inducting.  We're inducting on the variable $n$.  So the base case is $n=2$.

\[
X_2 = \{2\} \implies S = \{2\} \implies S_1 = \{2\} \implies P_1 =2.
\]

So the base case

\[
\sum_{i=1}^{1} P_i = 1 = \frac{3!}{2} - 1.
\]

So the base case is settled.  Now let's look at the inductive assumption.

\[
\sum_{i=1}^{2^{k-1}-1} P_i = \frac{(k+1)!}{2} -1
\]

When we go from $X_k$ to $_{k+1}$ we see that $S$ increases from having $2^{k-1}-1$ elements to $2^k-1$ elements.  For example if $k=11$ then $S$ goes from having 1023 elements to 2047 elements.  Basically, we've accounted for less than half of the elements.  So how to we account for all these elements in one go?  The ``easiest" answer is to write the sets $S_i$ in a particular order.  The order I'll choose is this:\\

Write all the elements for $X_k$ first, then the set $\{k+1\}$ by itself, then repeat the ordering from $X_k$ exactly and union $\{k+1\}$ to each element.  Therefore we have the following set of equations
\[
S_{i+2^{k-1}} = S_i \cup \{k+1\}
\] 

This makes computing the products much easier
\[
P_{i+2^{k-1}} = (k+1)P_i
\]


Now this makes the induction much easier

\[
\sum_{i=1}^{2^k - 1}P_i = \sum_{i=1}^{2^{k-1}-1}P_i + \sum_{i=2^{k-1}}^{2^k-1}P_i
\]
But we know by our set ordering and our assumption what these pieces are
\[
\sum_{i=1}^{2^k-1}P_i = \frac{(k+1)!}{2}-1 + (k+1) + (k+1)\cdot\sum_{i=1}^{2^{k-1}-1}P_i
\]

Before you look at other line of mathematics, make sure that you understand what all the indices are doing.  If you simply glanced at the previous line, chances are you didn't digest it, and the next line will seem like magic, which it most certainly is not.

\[
\sum_{i=1}^{2^k-1}P_i = \frac{(k+1)!}{2}-1 + (k+1) + (k+1)\left(\frac{(k+1)!}{2}-1\right)
\]

And we check
\[
\frac{(k+1)!}{2}-1 + (k+1) + (k+1)\left(\frac{(k+1)!}{2}-1\right) = \left(\frac{(k+2)!}{2}-1\right)
\]

which is exactly what we wanted.


\end{document}