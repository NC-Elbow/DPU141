\documentclass[16 pt]{amsart}
\usepackage{amscd,amsmath,amsthm,amssymb}
\usepackage{enumerate,varioref}
\usepackage{epsfig}
\usepackage{graphicx}
\usepackage{mathtools}
\newtheorem{thm}{Theorem}
\newtheorem{cor}[thm]{Corollary}
\newtheorem{lem}[thm]{Lemma}
\newtheorem{prop}[thm]{Proposition}
\theoremstyle{definition}
\newtheorem{defn}[thm]{Definition}
\theoremstyle{remark}
\newtheorem{ex}[thm]{Example}
\newtheorem{rem}[thm]{Remark}
\numberwithin{equation}{subsection}
\newcommand{\R}{\mathbb{R}}
\newcommand{\Z}{\mathbb{Z}}
\newcommand{\C}{\mathbb{C}}
\newcommand{\Q}{\mathbb{Q}}
\newcommand{\lh}{\lim_{h\rightarrow 0}}
\begin{document}

\title{Midterm Maths 141 Autumn 2015 \\ DePaul University\\Dr. Alexander}
\maketitle
You have 90 minutes to complete this exam.  Calculators are allowed, but no other electronic devices are permitted.  Please write all your answers in complete, legible sentences, and show all your work to receive full credit.  There are seven (7) problems here.  You may choose to do any six (6) of them.  
\vspace{1in}


%table
\begin{center}
  \begin{tabular}{ c | c }
    Problem & Score\\
    \hline
    &\\
    1&\\
    &\\
    2&\\
    &\\
    3&\\
    &\\
    4&\\
    &\\
    5&\\
    &\\
    6&\\
    &\\
    7&\\
    &\\
    Bonus&\\
    &\\
    \hline 
    &\\    
    Total& 
 \end{tabular}
\end{center}

\newpage 
Problem 1. Give the proper negation of the statement:
\[
\forall x\in\R, \exists y\in\Z \hspace{2mm} \text{such that} \hspace{2mm} y<\lfloor x^{2}e^{-x}\rfloor \text{ and } x>y.
\]

\vspace{1in}

Solution:

\[
\exists x\in\R \text{ such that } \forall y\in\Z, \hspace{2mm}  \hspace{2mm} y\ge \lfloor x^{2}e^{-x}\rfloor \text{ or } x\le y.
\]


\newpage
Problem 2.
Prove by induction
\[
\sum_{j=1}^n (-1)^j j^2 = \frac{(-1)^n n(n+1)}{2}
\]

\vspace{1in}

Solution: Let's first check the base case.
\[
\sum_{j=1}^1 (-1)^j j^2 = =-1 = \frac{(-1)^1 1(1+1)}{2}
\]
So this checks out.\\

For the inductive step let's assume for some $k\ge 1$ that 

\[
\sum_{j=1}^k (-1)^j j^2 = \frac{(-1)^k k(k+1)}{2}
\]
And let's see what happens when we add an additional term

\[
\sum_{j=1}^{k+1} (-1)^j j^2 = \sum_{j=1}^k (-1)^j j^2 + (-1)^{k+1}(k+1)^2 = \frac{(-1)^k k(k+1)}{2} + (-1)^{k+1}(k+1)^2
\]

Let's factor out a $(-1)^k(k+1)$ and we have
\[
(-1)^k (k+1) \left(\frac{k}{2} - (k+1)\right)
\]
Getting a common denominator and simplifying we have

\[
(-1)^k (k+1) \left(\frac{k-2(k+1)}{2}\right) = (-1)^{k+1}\frac{(k+1)(k+2)}{2}
\]
which is what we wanted to show.


\newpage
Problem 3.
Prove by induction
\[
\prod_{j=1}^{n} \frac{j+2}{j}= \frac{(n+1)(n+2)}{2} 
\]

\vspace{1in}

Solution: For the base case $(n=1)$ we have

\[
\prod_{j=1}^{1} \frac{j+2}{j}= 3 = \frac{(1+1)(1+2)}{2} 
\]
so this checks out.

Now for the inductive step let's assume for some $k\ge 1$ that
\[
\prod_{j=1}^{k} \frac{j+2}{j}= \frac{(k+1)(k+2)}{2} 
\]

And let's see what happens when we add an additional term.

\[
\prod_{j=1}^{k+1} \frac{j+2}{j}= \prod_{j=1}^{k} \frac{j+2}{j} \cdot \frac{k+3}{k+1}= \frac{(k+1)(k+2)}{2}\cdot \frac{k+3}{k+1}
\]

Canceling a $k+1$ gives the desired result.

\newpage

Problem 4.
Prove by induction
\[
3^n < n! \text{ for all } n>7.
\]

\vspace{1in}

Solution: For the base case, we check directly $3^8 < 8!$.  Aside note, $3^7 <7!$ but this was not the base case as written.

Now we assume the $3^k < k!$ for some $k\ge 8$ and then
\[
3^{k+1} = 3^k \cdot 3 < k! \cdot 3 < k!\cdot (k+1) = (k+1)!
\]

The second inequality is justified by the fact that 
\[
3 < 7 < k < k+1.
\]

\newpage 

Problem 5.
Prove by induction
\[
3 | (2^n - (-1)^n) \text{ for all } n>0.
\]

\vspace{1in}

Solution: For the base case $n=1$ we have
\[
3 | (2 - (-1))
\]
which is true.

Now assume for some $k>0$ that

\[
3 | (2^k - (-1)^k).
\]

Consider now that we can build $2^{k+1} - (-1)^{k+1}$ from $2^k - (-1)^k$ as follows.

\[
2(2^k - (-1)^k) + (-1)(2- (-1)) = 2^{k+1} - (-1)^{k+1}4
\]

We know that $3$ divides the first term in the sum by inductive hypothesis and $3$ divides the second term in the sum by the base case.  Therefore $3$ divides the entire sum.  or

\[
3 | (2^{k+1} - (-1)^{k+1})
\]

\newpage

Problem 6.
Verify that the sequence $a_n =(1+n) 2^n $ solves the following recurrence relation
\[
a_{n+2} - 4a_{n+1} + 4a_n = 0 
\]
with $a_0 =1$ and $a_1 = 4$

\vspace{1in}

Solution: Let's check the initial conditions first.  With $n=0$ we have 
\[
a_0 = (1+0)2^0 = 1
\]
and
\[
a_1 = (1+1)2^1 = 4
\]

So these check out.

Now let's check the general recurrence:

\[
a_{n+2} - 4 a_{n+1} + 4a_n = (1+n+2)2^{n+2} - 4(1+n+1)2^{n+1} + 4(1+n)2^n
\]

Let's factor out a $2^n$ and we see
\[
a_{n+2} - 4 a_{n+1} + 4a_n = 2^n [4(n+3)-8(n+2)+4(n+1)] = 0
\]

Verifying that this is zero is simple arithmetic.  Therefore the sequence $a_n = (1+n)2^n$ satisfies the recurrence relation.

\newpage

Problem 7. Let $A= \{a,b,c,d,e,f\}$ and define the relation $R$ on the power set $\mathcal{P}(A)$ as follows
\[
S_1 \textbf{R} S_2 \iff 2|S_1| \ge |S_2|^2
\]

Is the relation\\
(a) Reflexive?\\
(b) Symmetric?\\
(c) Transitive?

\vspace{1in}

Solution: This relation is not reflexive. Consider $S = \{a,b,c\}$ and so $|S|=3$ 
\[
2|S| = 6 < 9 = |S|^2
\]

So this set is not related to itself and then the relation is not reflexive.\\


(b) This relation is not symmetric.  Consider the two sets $S_1 = \{a,b\}$ and $S_2 = \{f\}$.  Then $|S_1|=2$, and $|S_2|=1$.

So
\[
2|S_1| = 4 \ge 1 = |S_2|^2
\]
And thus $S_1$ is related to $S_2$, however, 
\[
2|S_2| = 2 < 4 = |S_1|^2
\]
and so $S_2$ is not related to $S_1$.  Therefore the relation is not symmetric.\\

(c) This relation is, however, transitive.  Since the set is small enough, we only need to test the sizes of subsets.  We're only dealing with the possibilities $0,1,2,3,4,5,6$ There are no larger subsets than that.  From here we can actually show by exhaustion the relation is transitive.

Consider the pairs 
\[(0,0),(1,0),(2,0),(3,0),(4,0),(5,0),(6,0),\]
\[(1,1),(2,1),(3,1),(4,1),(5,1),(6,1),\]
\[(2,2),(3,2),(4,2),(5,2),(6,2),\]
\[(5,3),(6,3)
\]

This is the complete list of all relations within this set ordered by sizes of subsets.

By exhaustive search we can see that this holds.  In the integers at large, we could use a negative number to show that this relation is not transitive.  However, in our small context this relation on this particular power set is transitive.

\newpage

Bonus. Let $n,m\in\Z$ both be positive. Prove
\[
\prod_{k=1}^{n} \frac{k+m}{k}
\]
is an $m^{th}$ order polynomial.  What is this polynomial? Evaluate in terms of binomial coefficients when $n=m$ i.e.
\[
\prod_{k=1}^{n} \frac{k+n}{k}
\]

\end{document}