\documentclass[16 pt]{amsart}
\usepackage{amscd,amsmath,amsthm,amssymb}
\usepackage{enumerate,varioref}
\usepackage{epsfig}
\usepackage{graphicx}
\usepackage{mathtools}
\usepackage{tikz}
\usepackage{amsfonts}
\usepackage{svg}
\usetikzlibrary{graphs,arrows,topaths}
\newtheorem{thm}{Theorem}
\newtheorem{cor}[thm]{Corollary}
\newtheorem{lem}[thm]{Lemma}
\newtheorem{prop}[thm]{Proposition}
\theoremstyle{definition}
\newtheorem{defn}[thm]{Definition}
\theoremstyle{remark}
\newtheorem{ex}[thm]{Example}
\newtheorem{rem}[thm]{Remark}
\numberwithin{equation}{subsection}
\newcommand{\R}{\mathbb{R}}
\newcommand{\Z}{\mathbb{Z}}
\newcommand{\C}{\mathbb{C}}
\newcommand{\Q}{\mathbb{Q}}

\begin{document}

\title{Homework 7 Maths 141 Winter 2015}
\maketitle 


8.4.11: Prove by induction that if $a\equiv c \mod{n}$ then $a^m\equiv c^m \mod{n}$

\vspace{1in}

Solution:
The base case is the assumption that $a\equiv c\mod{n}$ For the inductive step assume for some $k$ that
\[
a^k \equiv c^k \mod{n}
\]

Then we have

\[
a^{k+1} \equiv a\cdot a^k \equiv a\cdot c^k \equiv c\cdot c^k \equiv c^{k+1} \mod{n}
\]

Where the second equivalence is our inductive hypothesis, the third equivalence is the original hypothesis, and the other steps are basic algebra.



\newpage

8.4.31: a. Find an inverse of $210 \mod{13}$.

\vspace{.5in}

The easiest way to deal with this (in my opinion) is to reduce 210 to 2.  ie $210\equiv 2\mod{13}$.  Then we consider that $2\cdot 7 = 14 \equiv 1\mod{13}$.  So $7$ is an inverse.

\vspace{.5in}

b. Find a positive inverse.\\
Solution: 7>0. 

\vspace{.5in}

c. Find an $x$ so that
\[
210x \equiv 8 \mod{13}
\]

Since $210\cdot 7 \equiv 1$ we have $(210 \cdot 7)\cdot 8 \equiv 8$ and since multiplication is associative we have
\[
210\cdot 56 \equiv 210 \cdot 4 \equiv 8 \mod{13}
\]

So this $x$ is 56 or $x\equiv 4\mod{13}$

\newpage


8.4.32: a. Find an inverse for $41\mod{660}$.

\vspace{.5in}

Solution: Consider $660 = 41\cdot 16 + 4$ 
So we can rewrite 
\[
4 = 660 - 41\cdot 16
\] 
Now we reduce 41
\[
41 = 4\cdot 10 + 1 \implies 1 = 41 - 4\cdot 10
\]

So we rewrite 
\[
1 = 41 - 4\cdot 10 = 41 - (660-41\cdot 16)\cdot 10 = 660 \cdot 10 + 161\cdot 41
\]

In this case we see that 
\[
41\cdot 161 = 1  + \text{ som multiple of } 660
\]
Which is to say
\[
41\cdot 161 \equiv 1 \mod{660}
\]

\vspace{.5in}

b. Find the least positive $x$ so that
\[
41x \equiv 125 \mod{660}
\]

\vspace{.5in}

Solution: We know from the previous problem that $41\cdot (161\cdot 125) \equiv 125$. Now we reduce this number
\[
125\cdot 161 \equiv 20125 \equiv 325 \mod{660}
\]

We know that this is the least positive value because it satisfies the equation and it is positive and less than $n$ where $n$ is the number of equivalence classes we're allowing.

\newpage


8.4.38:  Find the least positive inverse of $43 \mod{660}$

\vspace{1in}

Solution:We apply the same technique as before:
\[
660 = 43* 15 + 15
\]
\[
43 = 15 * 2 + 13 
\]
\[
15 = 13 * 1 + 2
\]
\[
13 = 2* 6 + 1
\]
\[
1 = 13- 2*6 = 13 - (15-13)* 6 = ... = 307*43 - 20*660
\]

So we arrive at
\[
43*307\equiv 1\mod{660}
\]

Or $307$ is the least positive inverse of $43\mod{660}$

\end{document}