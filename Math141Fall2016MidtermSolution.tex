\documentclass[16 pt]{amsart}
\usepackage{amscd,amsmath,amsthm,amssymb}
\usepackage{enumerate,varioref}
\usepackage{epsfig}
\usepackage{graphicx}
\usepackage{mathtools}
\newtheorem{thm}{Theorem}
\newtheorem{cor}[thm]{Corollary}
\newtheorem{lem}[thm]{Lemma}
\newtheorem{prop}[thm]{Proposition}
\theoremstyle{definition}
\newtheorem{defn}[thm]{Definition}
\theoremstyle{remark}
\newtheorem{ex}[thm]{Example}
\newtheorem{rem}[thm]{Remark}
\numberwithin{equation}{subsection}
\newcommand{\R}{\mathbb{R}}
\newcommand{\Z}{\mathbb{Z}}
\newcommand{\C}{\mathbb{C}}
\newcommand{\Q}{\mathbb{Q}}
\newcommand{\half}{\frac{1}{2}}
\newcommand{\lh}{\lim_{h\rightarrow 0}}
\begin{document}

\title{Midterm Maths 141 Fall 2016 \\ DePaul University\\Dr. Alexander}
\maketitle
You have 90 minutes to complete this exam.  Calculators are allowed, but no other electronic devices are permitted.  Please write all your answers in complete, legible sentences, and show all your work to receive full credit.  Please complete all seven (7) problems.


%table
\begin{center}
  \begin{tabular}{ c | c }
    Problem & Score\\
    \hline
    &\\
    1&\\
    &\\
    2&\\
    &\\
    3&\\
    &\\
    4&\\
    &\\
    5&\\
    &\\
    6&\\
    &\\
    7&\\
    &\\
    Bonus&\\
    &\\
    \hline 
    &\\    
    Total& 
 \end{tabular}
\end{center}

\newpage 
Problem 1.  Negate the following statement:
\[
\forall \epsilon>0, \exists \delta>0 \text{ s.t. If } 0<|x|<\epsilon \text{ then } |f(x)|<\delta
\]

\vspace{1in}

We know that the negation of a conditional becomes ``and not"
\[
\sim (p\rightarrow q)\equiv p\wedge \sim q
\]

So our proper negation is

\[
\exists \epsilon>0 \text{ s.t. } \forall \delta>0,  0<|x|<\epsilon \text{ and } |f(x)|\ge\delta
\]

\newpage

Problem 2. Prove by induction: For every $n\ge 1$.
\[
\sum_{j=1}^{n} \frac{1}{j(j+1)} = \frac{n}{n+1}
\]

\vspace{1in}

Solution: The base case is easy
\[
\sum_{j=1}^{1} \frac{1}{j(j+1)} = \frac{1}{1(1+1)} = \half
\]

Now for the inductive step, let's assume this works at the $k^{th}$ step


\[
\sum_{j=1}^{k} \frac{1}{j(j+1)} = \frac{k}{k+1}
\]
And let's show this works at the $k+1^{st}$ step.

\[
\sum_{j=1}^{k+1} \frac{1}{j(j+1)} = \frac{k}{k+1} + \frac{1}{k(k+1)}
\]

Let's factor out a common deminator

\[
\frac{1}{k+1}\left(k + \frac{1}{k+2} \right) = \frac{1}{k+1}\left(\frac{k(k+2)+1}{k+2}\right) = \frac{1}{k+1}\left(\frac{(k+1)^2}{k+2}\right)
\]

Canceling one power of $k+1$ we have

\[
\sum_{j=1}^{k+1} \frac{1}{j(j+1)} = \frac{k+1}{k+2} 
\]



\newpage
Problem 3.
Prove by induction: For every $n\ge 1$
\[
\begin{bmatrix}
1&1\\
1&1
\end{bmatrix}^{n+1} = 
\begin{bmatrix}
2^n & 2^n \\
2^n & 2^n
\end{bmatrix}
\]


\vspace{1in}

Solution: Again the base case is easy
\[
\begin{bmatrix}
1&1\\
1&1
\end{bmatrix}^2 = 
\begin{bmatrix}
2 & 2 \\
2 & 2
\end{bmatrix}
\]

Now suppose this works at the $k^{th}$ step

\[
\begin{bmatrix}
1&1\\
1&1
\end{bmatrix}^{k+1} = 
\begin{bmatrix}
2^k & 2^k \\
2^k & 2^k
\end{bmatrix}
\]

And show it works at the $k+1^{st}$.
\[
\begin{bmatrix}
1 & 1\\
1 & 1
\end{bmatrix}^{k+2} = 
\begin{bmatrix}
1 & 1\\
1 & 1
\end{bmatrix}^{k+1}
\begin{bmatrix}
1 & 1\\
1 & 1
\end{bmatrix} =
\begin{bmatrix}
2^k & 2^k \\
2^k & 2^k
\end{bmatrix}
\begin{bmatrix}
1 & 1\\
1 & 1
\end{bmatrix}
\]


Multiplying this out we get

\[
\begin{bmatrix}
2^k + 2^k & 2^k + 2^k\\
2^k + 2^k & 2^k + 2^k
\end{bmatrix} = 
\begin{bmatrix}
2^{k+1} & 2^{k+1}\\
2^{k+1} & 2^{k+1}
\end{bmatrix}
\]
which is what we wanted to show.

\newpage


Problem 4.
Prove by induction
\[
3^n < n! \text{ for all } n>6.
\]

\vspace{1in}

Solution: Again, the base case ($n=7$) is a direct verification
\[
3^7 = 2187 < 5040 = 7!
\]


Now the inductive case in this particular proof is relatively easy
\[
3^{k+1} = 3\cdot 3^k < 3\cdot k! < (k+1)k! = (k+1)!
\]

The first inequality is the inductive assumption.  The second inequality takes into account the fact that the base case is at $n=7$ and so
\[
k+1 > 7 > 3.
\]

\newpage 

Problem 5.
Prove by induction
\[
3 | (n^3-n) \text{ for all } n>0.
\]

\vspace{1in}

Solution: Again the base case is straight forward
\[
3| (1-1) 
\]
This is true since $0/3\in \Z$.

Now we assume 
\[
3|(k^3-k)
\]

Let's first expand $(k+1)^3-(k+1)$ we get
\[
(k+1)^3-(k+1) = (k^3-k) + 3(k^2+k)
\]

We assume that three divides the first piece and we know for certain that three divides the second piece (as it is written explicitly as a multiple of 3).  Since three divides each piece it divides their sum

\[
3| (k+1)^3-(k+1)
\]
which is what we wanted to show.

\newpage

Problem 6.
Solve the following recurrence relation:
\[
a_{n+2}+2a_{n+1}-2a_n = 0
\]
with initial conditions $a_0=0$ and $a_1=3$.


\vspace{1in}


Solution:
Using our theorem on second order linear homogeneous recurrence relations with constant coefficients, we assume:
\[
a_n = Cr^n
\]

This leaves us with the characteristic equation
\[
r^2 + 2r - 2 =0
\]

Solving for $r$ we have
\[
r_{\pm} = -1 \pm \sqrt{3}
\]

Now our general solution is
\[
a_n = C(-1+\sqrt{3})^n + D(-1-\sqrt{3})^n
\]

Plugging in our initial conditions:
\begin{eqnarray}
a_0 = 0 & = & C + D \nonumber \\
a_1 = 3 & = & C(r_+)+D(r_-) \nonumber
\end{eqnarray}

We can solve these as
\[
C = \frac{\sqrt{3}}{2}, \text{ and } D = \frac{-\sqrt{3}}{2}
\]

Which yields the specific solution

\[
a_n = \frac{\sqrt{3}}{2}(-1+\sqrt{3})^n - \frac{\sqrt{3}}{2}(-1-\sqrt{3})^n
\]

\newpage

Problem 7. Let $A=\Z$ and consider the relation $R$ on $A$.
\[
nRm \iff 3|(n+m)
\]


For each of the following questions,if the property holds, prove it. If not give a counterexample.\\

(a) Is $R$ reflexive?\\

(b) Is $R$ symmetric?\\

(c) Is $R$ transitive?


\vspace{1in}

Solution: (a) This relation is not reflexive since $(1,1)\notin R$
\[
3 \not| (1+1)
\]


(b) This relation is, however, symmetric since
\[
(n,m)\in R \iff 3 | (n+m) \iff 3|(m+n) \iff (m,n)\in R
\]

(c) This relation is not transitive since we have the triple $(a,b,c)=(1,2,1)$ where
\[
(a,b)\in R \wedge (b,c)\in R \text{ but } (a,c)\notin R
\]

That is
\[
3|(1+2) \wedge 3|(2+1) \wedge 3\not|(1+1)
\]

\newpage

Bonus. Show 
\[
7 | (2222^{5555} + 5555^{2222}).
\]

This requires a little modular arithmetic
\begin{eqnarray}
7 | (2222^{5555} + 5555^{2222}) & \iff & 7|(3^{5555} + 4^{2222}) \nonumber\\
& \iff & 7|(3^5 + 4^2)\nonumber\\
& \iff & 7|(5+2) \nonumber
\end{eqnarray}

The first $\iff$ is justified by the fact that 2222 has the same remainder as 3, and 5555 has the same remainder as 4.  Thus raising these remainders to higher powers doesn't change divisibility on 7.  The second $\iff$ is justified by the fact that
\[
7| a^7 - a \implies a^6 \text{ has remainder } 1.
\]

So every 6 powers we reset the remainder count to 1.  The third $\iff$ is justified by simple arithmetic.



\end{document}