\documentclass[16 pt]{amsart}
\usepackage{graphicx}
\usepackage{mathtools}
\newtheorem{thm}{Theorem}
\newtheorem{cor}[thm]{Corollary}
\newtheorem{lem}[thm]{Lemma}
\newtheorem{prop}[thm]{Proposition}
\theoremstyle{definition}
\newtheorem{defn}[thm]{Definition}
\theoremstyle{remark}
\newtheorem{ex}[thm]{Example}
\newtheorem{rem}[thm]{Remark}
\numberwithin{equation}{subsection}
\newcommand{\R}{\mathbb{R}}
\newcommand{\Z}{\mathbb{Z}}
\newcommand{\C}{\mathbb{C}}
\newcommand{\Q}{\mathbb{Q}}
\newcommand{\lh}{\lim_{h\rightarrow 0}}
\begin{document}

\title{Homework 7 Maths 141 Spring 2014}
\maketitle 

11.2.9: Write the following with $O, \Omega,$ or $\Theta$ notation.\\
\[
\frac{1}{2}x^2 \leq |3x^2-80x+7| \leq 3|x^2|
\]
For all real numbers $x>25$.


This follows the definition of $\Theta$
\[
f(x) \text{ is } \Theta(g(x)) \text{ if } \exists A,B,N>0 \text{ so that } A|g(x)| \leq |f(x)| \leq B|g(x)| \forall x>N
\]


Thus 
\[
3x^2-80x+7 \text{ is } \Theta(x^2)
\]

\newpage

11.2.31: Refer the the results of earlier problems to write $O,\Omega,$ or $\Theta$ notation for $7x^4-95x^3+3$.

Earlier exercises indicate that

\[
7x^4-95x^3+3 \text { is } \Omega(x^4).
\]
Since 
\[
|x^4|\leq |7x^4-95x^3+3|, \forall x> 16
\]
and
\[
7x^4-95x^3+3 \text { is } O(x^4)
\]
since
\[
|7x^4-95x^3+3|\leq 105 |x^4|.
\]
therefore
\[
7x^4-95x^3+3 \text { is } \Theta(x^4).
\]




\newpage

11.2.58: Use mathematical induction to prove\\

a. 
\[
1^{1/3} + 2^{1/3} + \dots + n^{1/3} \leq n^{4/3}
\]
for all integers $n>0$.\\


First of all, we check the base case:
\[
1^{1/3} = 1 \leq 1^{4/3} =1.
\]
and so this checks out.
Now for the inductive step:  Assume
\[
1^{1/3} + 2^{1/3} + \dots + k^{1/3} \leq k^{4/3}
\]
We must show
\[
1^{1/3} + 2^{1/3} + \dots + (k+1)^{1/3} \leq (k+1)^{4/3}
\]
Here we have 
\begin{eqnarray*}
\sum_{j=1}^{k+1} j^{1/3}& =& \sum_{j=1}^{k} j^{1/3} + (k+1)^{1/3}\\
& \leq & k^{4/3} + (k+1)^{1/3}\\
& \leq & k\cdot k^{1/3} + (k+1)^{1/3}\\
& \leq & k\cdot (k+1)^{1/3} + (k+1)^{1/3}\\
& = & (k+1)\cdot (k+1)^{1/3}\\
& = & (k+1)^{4/3} 
\end{eqnarray*}


And thus
\[
1^{1/3} + 2^{1/3} + \dots + n^{1/3} \leq n^{4/3}
\]

or
\[
1^{1/3} + 2^{1/3} + \dots + n^{1/3} \text{ is } O(n^{4/3}).
\]

\vspace{1in}

b. Prove
\[
\frac{1}{2}n^{4/3} \leq 1^{1/3} + 2^{1/3} + \dots + n^{1/3} 
\]

Again, the base step is nearly trivial
\[
\frac{1}{2} 1^{4/3} = \frac{1}{2} < 1^{1/3} =1.
\]

Now for the inductive step, we assume
\[
\frac{1}{2}k^{4/3} \leq 1^{1/3} + 2^{1/3} + \dots + k^{1/3} 
\]
we want to show
\[
\frac{1}{2}(k+1)^{4/3} \leq 1^{1/3} + 2^{1/3} + \dots + (k+1)^{1/3} 
\]

So we have 
\begin{eqnarray*}
\sum_{j=1}^{k+1} j^{1/3} & = & \sum_{j=1}^{k} j^{1/3} + (k+1)^{1/3} \\
          & \geq & \frac{1}{2}k^{4/3} + (k+1)^{1/3}\\  
\end{eqnarray*}

From here we have to step aside for a moment and see why this last step moves us toward our goal.
Consider
\[
k^2 > k^2-1 = (k+1)(k-1)
\]
This means
\[
\frac{k}{k-1} > \frac{k+1}{k} > \left(\frac{k+1}{k}\right)^{1/3}
\]
Now we cross multiply and see
\[
k\cdot k^{1/3} > (k-1)\cdot (k+1)^{1/3}
\]
From here we apply the trick of adding zero and see
\[
(k-1)(k+1)^{1/3} = (k+1-2)(k+1)^{1/3} = (k+1)^{4/3} - 2(k+1)^{1/3}
\]

We divide by two and look back to the original inequality
\[
k^{4/3} > (k+1)^{4/3} - 2(k+1)^{1/3}
\]

Rearranging we see
\[
\frac{1}{2} k^{4/3} + (k+1)^{1/3} \geq \frac{1}{2}(k+1)^{4/3}
\]
which is exactly where we left our proof by induction.
Thus
\begin{eqnarray*}
\sum_{j=1}^{k+1} j^{1/3} & = & \sum_{j=1}^{k} j^{1/3} + (k+1)^{1/3} \\
          & \geq & \frac{1}{2}k^{4/3} + (k+1)^{1/3}\\  
          & \geq & \frac{1}{2}(k+1)^{4/3}     
\end{eqnarray*}

And so
\[
1^{1/3} + 2^{1/3} + \dots + n^{1/3} \text{ is } \Omega(n^{4/3})
\]

\vspace{1in}

c. Putting (a) and (b) together we see
\[
\frac{1}{2}n^{4/3} \leq 1^{1/3} + 2^{1/3} + \dots + n^{1/3} \leq n^{4/3}
\]

or 
\[
1^{1/3} + 2^{1/3} + \dots + n^{1/3} \text{ is } \Theta(n^{4/3}).
\]

\newpage


4.8.28: Prove for all integers $a,b$ that $gcd(a,b)|lcm(a,b)$.

This problem could not be more simple.  Since $gcd(a,b)|a$ and $a|lcm(a,b)$ we can write
\[
a = d\cdot gcd(a,b), \text{ and } a\cdot \ell = lcm(a,b)
\]
where $d,\ell$ are integers.  Then
\[
lcm(a,b) = a\cdot \ell = gcd(a,b)\cdot d\cdot \ell
\]
and so
\[
gcd(a,b) | lcm(a,b).
\]

\newpage


11.3.19: Count the number of operations for the given pseudocode.\\

$\begin{array}{llll}
\textbf{for} & i = 1 \text{ to } n& &\\
             & \textbf{for} & j=1 \text{ to } i & \\
             &    & \textbf{for} & k=1 \text{ to } j\\
             &    &       & x:= i\cdot j \cdot k\\  
             &   & \text{next } k & \\
             & \text{next } j & & \\
 \text{next } i & & & 
 \end{array}$      
 
 
 Looking to the inner most loop we have $j$ loops per sequence.  $j$, however, is incrementing relative to $i$.
 So if we only look at the inner two loops we see the there are 1,2,3,4,5,6 loops incrementing with $i$.
 Adding these we get
 \[
 1+2+3+\dots + i = \binom{i+1}{2}
 \]
 
 For each loop in $n$ we get $\binom{i+1}{2}$ additional loops.
 Thus, letting $\ell_n$ be the number of loop when $i=n$ we have the following recurrence relation;
 \[
 \ell_n = \ell_{n-1} + \binom{n+1}{2}
 \]
 In each loop are two multiplications and thus we have $2\ell_n$ total number of operations.
 
 In this case we can solve $\ell_n$ directly and see
 \[
 \ell_n = \sum_{m=2}^{n} \binom{n+1}{2} = \sum_{m=2}^{n} \frac{1}{2}(m^2+m) =\frac{1}{2}\left( \frac{n(n+1)(2n+1)}{6}-1 + \frac{n^2+n}{2}-1\right)
 \]
 
 We see that
 \[
 \ell_n = \frac{1}{12}n^3 + \text{ lower order terms } 
 \]
 and thus
 \[
 \ell_n \text{ is } \Theta(n^3)
 \]
 by the theorem of polynomial orders.  Since the number of operations is twice the number of loops, it too, is $\Theta(n^3)$.
      
\end{document}