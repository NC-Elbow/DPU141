\documentclass[10 pt]{amsart}
\usepackage{amscd,amsmath,amsthm,amssymb}
\usepackage{enumerate,varioref}
\usepackage{epsfig}
\usepackage{graphicx}
\usepackage{mathtools}
\newtheorem{thm}{Theorem}
\newtheorem{cor}[thm]{Corollary}
\newtheorem{lem}[thm]{Lemma}
\newtheorem{prop}[thm]{Proposition}
\theoremstyle{definition}
\newtheorem{defn}[thm]{Definition}
\theoremstyle{remark}
\newtheorem{ex}[thm]{Example}
\newtheorem{rem}[thm]{Remark}
\numberwithin{equation}{subsection}
\newcommand{\R}{\mathbb{R}}
\newcommand{\Z}{\mathbb{Z}}
\newcommand{\C}{\mathbb{C}}
\newcommand{\Q}{\mathbb{Q}}
\newcommand{\lh}{\lim_{h\rightarrow 0}}
\begin{document}

\title{Homework 2 Maths 141 Spring 2014}
\maketitle 

Problem 5.1.42: Write the following as a single summation:
\[
\sum_{m=0}^{n} (m+1)2^m   + (n+2)2^{n+1}
\]

Solution: We see that the form of the summands is simply $(k+1)2^k$.  The additional summand in the above equation follows this form and therefore we simply increase the ending index by one, thus:
\[
\sum_{m=0}^{n+1} (m+1)2^m  
\]
 


\newpage

Problem 5.2.16: Use mathematical induction to prove the following formula:
\[
\left(1-\frac{1}{2^2}\right)\left(1-\frac{1}{3^2}\right)\cdots\left(1-\frac{1}{n^2}\right) = \frac{n+1}{2n}
\]
for all integers $n\geq 2$.\\


Solution:
\begin{proof}
We will prove this by mathematical induction.  To simplify things let's use product notation rather than writing out a full product.  Let $P(n)$ be the predicate
\[
\prod_{j=2}^{n} \left(1-\frac{1}{j^2}\right) = \frac{n+1}{2n}.
\]
To establish the base step ($P(2)$) we see
\[
\prod_{j=2}^{2} \left(1-\frac{1}{j^2}\right) = 1-1/4=3/4 = \frac{2+1}{2(2)}.
\]
Thus we have verified the base step $P(2)$.\\
Now to establish the inductive step, we assume the predicate $P(k)$ to be true and logically deduce that $P(k+1)$ must also be true.\\
$P(k)$ is the predicate
\[
\prod_{j=2}^{k} \left(1-\frac{1}{j^2}\right) = \frac{k+1}{2k}.
\]
So we begin with the predicate $P(k+1)$
\begin{eqnarray*}
\prod_{j=2}^{k+1} \left(1-\frac{1}{j^2}\right) &=& \prod_{j=2}^{k} \left(1-\frac{1}{j^2}\right) \cdot \left(1-\frac{1}{(k+1)^2}\right)\\
&=& \frac{k+1}{2k}\cdot \left(1-\frac{1}{(k+1)^2}\right)\\
&=& \frac{k+1}{2k}\cdot \left(\frac{(k+1)^2-1}{(k+1)^2}\right)\\
 &=& \frac{k+1}{2k}\cdot \left(\frac{k(k+2)}{(k+1)^2}\right)\\
 &=& \frac{k+2}{2(k+1)}
\end{eqnarray*}
Which is what we wished to show.
\end{proof}

\newpage

Problem 5.2.35: Find the error in the ``proof" by induction:
\begin{lem}
For any integer $n\geq 1$
\[
\sum_{i=1}^{n} i(i!) = (n+1))! -1
\]
\end{lem}

\begin{proof}
Let $P(n)$ be the predicate
\[
\sum_{i=1}^{n} i(i!) = (n+1))! -1
\]
We show $P(1)$
\[
\sum_{i=1}^{1} i(i!) = 1(1!) = 1 = (1+1))! -1
\]
Therefore $P(1)$ is true.
\end{proof}

Solution:  Obvioiusly this proof is missing the inductive step.  While the base step is shown to be true in the correct way, there are infinitely formulas that could satisfy the base step without adequately satisfying the inductive step.  For example
\[
\sum_{i=1}^{n} i(i!) = n^2-n+1
\]
or
\[
\sum_{i=1}^{n} i(i!) = n^n
\]
or 
\[
\sum_{i=1}^{n} i(i!) = 2^{2n}-n.
\]
Each of these certainly satisfy the base step because using the evaluation at $n=1$ we see the right hand sides all yield 1.  We want to show the inductive step to be true as well.  So let's finish the proof.\\

Let $P(k)$ be the predicate
\[
\sum_{i=1}^{k} i(i!) = (k+1))! -1
\]
then
\begin{eqnarray*}
\sum_{i=1}^{k+1} i(i!) &=& \sum_{i=1}^{k} i(i!) + (k+1)(k+1)!\\
&=& (k+1)!-1 + (k+1)(k+1)!\\
&=& (1+(k+1))(k+1)! - 1\\
&=& (k+2)(k+1)!-1 = (k+2)!-1 
\end{eqnarray*}




\newpage
Problem 5.3.13: Prove the following statement by mathematical induction:\\
For any integer $n\geq 0$, $x-y$ divides $x^n-y^n$.\\

Solution:  For this problem we need to be a little stealthy.  We will assume not only $P(k)$, but also $P(k-1)$ to be true and then show $P(k+1)$ to be true.  First, let's notice that if $a|b$ and $a|c$ then $a|(b+c)$.  Additionally, if $a|b$ then $a|kb$ for any integer $k$.\\

So we begin by trying to construct an integer multiple of $x^{k+1}-y^{k+1}$ using lower powers of $x^k-y^k$.
\[
x(x^k-y^k) + y(x^k-y^k) = x^{k+1}-y^{k+1} - xy(x^{k-1}-y^{k-1}).
\]
Therefore
\[
x^{k+1}-y^{k+1}= (x+y)(x^k-y^k) + xy(x^{k-1}-y^{k-1}).
\]
Since we have assumed 
\[
(x-y)|(x^{k-1}-y^{k-1})
\]
 by $P(k-1)$ and
\[
(x-y)|(x^{k}-y^{k})
\] by $P(k)$,

we see that a sum of integer multiples of these factors yields $x^{k+1}-y^{k+1}$ and thus we achieve $P(k+1)$.

\newpage
Problem 5.3.21: Prove the following statement by mathematical induction:\\
\[
\sqrt{n} <  \frac{1}{\sqrt{1}} + \frac{1}{\sqrt{2}} + \dots + \frac{1}{\sqrt{n}}
\]

Solution:  For this problem we simply need a little algebra.  Consider then
\begin{eqnarray*}
\sqrt{k}         & < & \sqrt{k+1}\\
\sqrt{k}\sqrt{k} & < & \sqrt{k}\sqrt{k+1}\\
k & < & \sqrt{k}\sqrt{k+1}\\
k+1 & < & \sqrt{k}\sqrt{k+1} +1\\
\frac{k+1}{\sqrt{k+1}} & < & \frac{\sqrt{k}\sqrt{k+1} +1}{\sqrt{k+1}} \\
\sqrt{k+1} & < & \sqrt{k} + \frac{1}{\sqrt{k+1}}
\end{eqnarray*}

Now that we have that matter settled let's begin the proof.
\begin{proof}
Let $P(k)$ be the predicate
\[
\sqrt{k} <  \frac{1}{\sqrt{1}} + \frac{1}{\sqrt{2}} + \dots + \frac{1}{\sqrt{k}}.
\]
We will prove $P(n), n\geq2$ by induction.\\
Base step: $P(2)$:
\[
\sqrt{2} < \frac{1}{\sqrt{1}} + \frac{1}{\sqrt{2}}
\]
We can verify this by direct computation since $\sqrt{2}<1.5< \frac{1}{\sqrt{1}} + \frac{1}{\sqrt{2}} \approx 1.7$ so the base step is established.\\

Now for the inductive step:  Assume $P(k)$ to be true then by the algebra above we see
\[
\sqrt{k+1} < \sqrt{k} + \frac{1}{\sqrt{k+1}} < \left(\frac{1}{\sqrt{1}} + \frac{1}{\sqrt{2}} + \dots + \frac{1}{\sqrt{k}}\right) + \frac{1}{\sqrt{k+1}} 
\]
Which is what we wish to show.\\
\end{proof}
\begin{rem}
This is roughly equivalent to the Riemann sum for
\[
\sqrt{x} = c \int x^{-1/2} dx.
\]
\end{rem}
\end{rem}
\end{document}