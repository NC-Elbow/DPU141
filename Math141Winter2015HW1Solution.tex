\documentclass[16 pt]{amsart}
\usepackage{amscd,amsmath,amsthm,amssymb}
\usepackage{enumerate,varioref}
\usepackage{epsfig}
\usepackage{graphicx}
\usepackage{mathtools}
\usepackage{svg}
\newtheorem{thm}{Theorem}
\newtheorem{cor}[thm]{Corollary}
\newtheorem{lem}[thm]{Lemma}
\newtheorem{prop}[thm]{Proposition}
\theoremstyle{definition}
\newtheorem{defn}[thm]{Definition}
\theoremstyle{remark}
\newtheorem{ex}[thm]{Example}
\newtheorem{rem}[thm]{Remark}
\numberwithin{equation}{subsection}
\newcommand{\R}{\mathbb{R}}
\newcommand{\Z}{\mathbb{Z}}
\newcommand{\C}{\mathbb{C}}
\newcommand{\Q}{\mathbb{Q}}
\newcommand{\lh}{\lim_{h\rightarrow 0}}
\begin{document}

\title{Homework 1 Maths 140 Winter 2015}
\maketitle 


3.3.24. Use laws of negation to derive the following rules:
a.

\[
\sim (\forall x \in D ( \forall y \in E (P(x,y))) \equiv \exists x\in D (\exists y\in E( \sim P(x,y)))
\]

\vspace{1in}

Solution: As we've seen, negation flips $\forall$ with $\exists$ and changes the places of `such that' with the comma. So 

\[
\sim (\forall x \in D ( \forall y \in E (P(x,y))) 
\]
needs $\exists x\in D$ and then $\exists y\in E$ and the simple negation $\sim P(x,y)$

which gives the result
\[
\exists x\in D (\exists y\in E( \sim P(x,y)))
\]

\vspace{1in}

b.
\[
\sim (\exists x\in D ( \exists y\in E( P(x,y))) \equiv \forall x\in D(\forall y\in E (\sim P(x,y)))
\]


\vspace{1in}

Solution:  Taking our rules from before

\[
\sim (\exists x\in D ( \exists y\in E( P(x,y))) 
\]

becomes
\[
\forall x\in D (\exists y\in E, (\sim P(x,y)))
\]


\newpage


4.3.29. For all $a,b$ if $a|b$ then $a^2|b^2$

\vspace{1in}

Solution: Beginning with the hypothesis $a|b$ we see $\exists k \in \mathbb{Z}$ so that $b = ka.$
Therefore
\[
b^2 = (ka)^2 = k^2 a^2 \implies a^2 | b^2
\]
 
since $k\in\Z$ and integers are closed under multiplication $k^2\in\Z$ and so $a^2$ divides $b^2.$



\newpage

4.3.41.How many zeroes are at the end of $45^8 \times 88^5$?


\vspace{1in}

Solution: We should observe that when can count the number of zeroes at the end of a number by writing a number in the form
\[
\text{Thing not divisible by ten} \times 10^n
\]

Then $n$ will be the number of zeroes at the end of the number.  So we should simply write $45^8 \times 88^5$ in it's prime factorization and then collect tens.

\[
45^8 \times 88^5 = (3\cdot 3\cdot 5)^8 \cdot (2\cdot 2\cdot 2\cdot 11)^5 = 2^{15}3^{16}5^8 11^5 = 2^{7}3^{16}11^5 (2\cdot 5)^8
\]

Since we have $1o^8$ in this number and no other divisors of ten, this number ends in eight zeroes.


\newpage

4.3.47. Observe that 
\begin{eqnarray*}
7524 &=& 7 \cdot 1000 + 5\cdot 100 + 2\cdot 10 + 4\\
        &=& 7\cdot(999+1) + 5 ...\\
&=& \text{an integer divisible by } 9 + (7+5+2+4)
\end{eqnarray*}

Generalize this to any number.

\vspace{1in}

Solution: We start with the recollection of the additive property of divisibility, that is:
\[
(a|b) \wedge (a|c) \implies a|(b+c)
\]


In practice this means:
\[
9 | b  \leftrightarrow 9 | (9k + b)
\]

So adding any multiple of 9 to $b$ does not change the divisibilty.

Here are some examples:
\[
9 | 18  \implies 9 | (18+999)
\]
More explicitly
\[
\forall n\geq 0, \hspace{.1in} 9 | (10^n -1).
\]

Now recalling the multiplicative property of divisibility:
\[
\forall c\in\Z, (a|b) \rightarrow (a|bc).
\]

So the practical implication is that
\[
a | b  \implies a|(10^n\cdot b)
\]

And in the case of nines we can write any number as
\[
d_nd_{n-1}d_{n-2}\dots d_1d_0 = d_n\cdot 10^n + \cdots + d_1\cdot 10 + d_0
\]

Since $9|d_n \leftrightarrow 9|10^n d_n$ we have
\[
9|(\sum_{k=0}^{n} d_k\cdot 10^k) \longleftrightarrow 9| (\sum_{k=0}^{n} d_k)
\]

Which is the result we wished to show.


\newpage

4.4.37: The square of any integer is of the form $4k$ or $4k+1.$


\vspace{1in}


Solution: Here, we will use a proof by division into classes.
The statement $r=$`is of the form $4k$ or $4k+1$' is our conclusion.
So we let\\
 $p=$ `$n$ is an even integer,'\\
  and\\ 
  $q= `n$ is an odd integer.'

We will show
\begin{center}
$p\vee q$\\
$p\rightarrow r$\\
$q\rightarrow r$\\
$\therefore r$
\end{center}

All integers are even or odd, therefore $p\vee q$ is true.
Let's check the case when $n$ is even.\\
Since $n$ is even $\exists \ell\in\mathbb{Z}$ s.t. $n=2\ell$.\\
\[
n^2 = (2\ell)^2 = 4\ell^2 = 4 (k) 
\]
where $k=\ell^2$. So we have shown $p\rightarrow r$.\\

Now let's check when $n$ is odd.\\
Since $n$ is odd, $\exists j\in\Z$ s.t. $n=2j+1.$

\[
n^2 = (2j+1)^2 = 4j^2+4j+1 = 4(j^2+j) + 1 = 4k+1
\]
where $k=(j^2+j)$.  So we have shown $q\rightarrow r$.
Thus the overall statment is true.



\newpage

4.4.38. For any integer $n$ then $n^2+5$ is not divisible by 4.

\vspace{1in}

Solution: Let's again check this for even and odd integers.\\

When $n$ is even, there is some integer $k$ so that $n=2k$.
\[
n^2 + 5 = (2k)^2 + 5 = 4k^2 + 5 = 4(k^2+1) + 1.
\]

Clearly this is not divisible by 4, since there is a remainder of 1.\\

Now let's check when $n$ is odd.  Since $n$ is odd, we have some integer $\ell$ so that $n=2\ell+1$.

\[
n^2+5 = 4\ell^2 + 4\ell + 1+ 5 = 4(\ell^2+ \ell + 1) + 2
\]
Again, this is not divisible by four because we have a remainder of two.




\end{document}